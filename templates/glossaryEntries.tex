\makenoidxglossaries
\newglossaryentry{Sensor}
{
	name=Sensor,
	plural=Sensoren,
	description={Gerät zur Messung von physikalischen oder chemischen Eigenschaften},
}
\newglossaryentry{Cookie}
{
	name=Cookie,
	plural=Cookies,
	description={Speichert Textinformationen einer Website auf dem Computer des Nutzers},
}
\newglossaryentry{CookieNotice}
{
	name=Cookie Notice,
	description={Ein Hinweis, dass die Seite Cookies verwendet. Vorgeschrieben nach EU-Recht},
}
\newglossaryentry{Schadstoffwert}
{
	name=Schadstoffwert,
	plural=Schadstoffwerte,
	description={Schadstoffwerte geben die Anzahl von Schadstoffen in der Luft an. Dabei handelt es sich um Stoffe die schädlich für Mensch, Tiere oder Pflanzen sind},
}
\newglossaryentry{Lokalisierungsdatei}
{
	name=Lokalisierungsdatei,
	plural=Lokalisierungsdateien,
	description={Beinhalten Texte, die Platzhalter im Programm ersetzen. So können leicht alternative Übersetzungen eingesetzt werden},
}
\newglossaryentry{SmartAQnet}
{
	name=SmartAQnet,
	description={Das Smart Air Quality Network ist ein Projekt, dass den Aufbau eines intelligenten, reproduzierbaren Messnetzwerkes in der Modellregion Augsburg zum Ziel hat. Dabei werden bereits vorhandene Sensoren integriert, aber auch preiswerte neue Messtechnologien entwickelt und eingesetzt},
}
\newglossaryentry{Metadaten}
{
	name=Metadaten,
	description={Metadaten sind strukturierte Daten, die Informationen über Informationsressourcen wie zum Beispiel \glspl{Sensor} beinhalten},
}
\newglossaryentry{Kartenoverlay}
{
	name=Kartenoverlay,
	plural=Kartenoverlays,
	description={Meist farbliche Überlagerungen auf der Karte, die genutzt werden um Informationen darzustellen},
}
\newglossaryentry{Sensoroverview}
{
	name=Sensoroverview,
	plural=Sensoroverviews,
	description={Eine Übersicht über die \gls{Sensor} \gls{Metadaten} und die aktuell und in der Vergangenheit gemessenen \glspl{Schadstoffwert}},
}
\newglossaryentry{Interpolation}
{
	name=Interpolation,
	description={Verfahren zur kontinuierlichen Darstellung von diskreten Messwerten. Aus diskreten Messwerten wird dazu eine stetige Funktion abgeleitet},
}
\newglossaryentry{Git}
{
	name=Git,
	description={Freie Software zur verteilten Versionsverwaltung. Git unterstützt unter anderem das speichern von \glspl{Repository} auf Servern},
}
\newglossaryentry{GitHub}
{
	name=GitHub,
	description={Großer Anbieter für online-verwaltete \gls{Git}-\glspl{Repository}},
}
\newglossaryentry{Eclipse}
{
	name=Eclipse,
	description={Eine Entwicklungsumgebung die auf die Programmiersprache Java ausgelegt ist},
}
\newglossaryentry{Smartphone}
{
	name=Smartphone,
	plural=Smartphones,
	description={Ein Mobiltelefon mit umfangreichen Computer-Funktionalitäten und Konnektivitäten}
}
\newglossaryentry{IntelliJ}
{
	name=IntelliJ IDEA,
	description={IntelliJ IDEA ist eine integrierte Entwicklungsumgebung (IDE) des Softwareunternehmens JetBrains für die Programmiersprachen Java, Kotlin, Groovy und Scala}
}
\newglossaryentry{NetBeans}
{
	name=NetBeans IDE,
	description={Netbeans ist eine vollständig in Java geschriebene Entwicklungsumgebung, die hauptsächlich auf die Entwicklung von Java ausgelegt ist},
}
\newglossaryentry{Repository}
{
	name=Repository,
	plural=Repositories,
	description={Ein verwaltetes Verzeichnis zur Speicherung und Beschreibung von digitalen Objekten. In Git bilden Projekte Repositories}
}
\newglossaryentry{Maven}
{
	name=Maven,
	description={Ein Tool zur verwaltung von Java-Projekten. Maven wird von der Apache Software Foundation entwickelt. Durch die standardisierte Verzeichnisstruktur erlaubt Maven die Verwendung von verschiedenen Entwicklungsumgebungen für das selbe Projekt. Maven verwaltet auch Bibliotheken von Drittanbietern und vereinfacht damit das importieren von zusätzlichen Funktionalitäten},
}
\newglossaryentry{Standard-PC}
{
	name=Standard-PC,
	plural=Standard-PCs,
	description={Computer mit aktueller Hard- und Software. Aktuell heißt in diesem Fall, nicht älter als 5 Jahre}
}

\newglossaryentry{DIY}
{
	name=DIY,
	description={Eine Phrase aus dem Englischen und bedeutet übersetzt Mach es selbst. Mit der Phrase werden grundsätzlich Tätigkeiten bezeichnet, die von Amateuren ohne professionelle Hilfe ausgeführt werden}
}
\newglossaryentry{Feinstaub}
{
	name=Feinstaub,
	description={Bezeichnet eine Teil des Staubs in der Luft, welcher sehr feine Partikelgrößen aufweist. Als Luftschadstoff wirkt sich Feinstaub negativ auf die Gesundheit aus, so kausal auf die Sterblichkeit, Herz-Kreislauf-Erkrankungen und Krebserkrankungen sowie wahrscheinlich kausal auf Atemwegserkrankungen}
}
\newglossaryentry{Endgeraet} {
	name=Endgerät,
	plural=Endgeräte,
    description={Das genutzte Gerät um auf die Schnittstelle zuzugreifen},
}

\newglossaryentry{Luftqualitaetsdaten}{
    name=Luftqualitätsdaten,
	description={Die Daten, welche von den Benutzern abgefragt werden koennen. Darunter fallen Feinstaubdaten, Luftdruck, Luftfeuchtigkeit und Temperatur}
}
\newglossaryentry{Standard-Hardware}
{
	name=Standard-Hardware,
	description={Hardware in technischen Geräten zum Zeitpunkt der Entwicklung der Software nicht älter als 5 Jahre ist},
}
\newglossaryentry{Standard-Software}
{
	name=Standard-Software,
	description={Software, die zum Zeitpunkt der Entwicklung der Software nicht älter als 5 Jahre ist},
}
\newglossaryentry{Standard-Betriebssystem}
{
	name=Standard-Betriebssystem,
	plural=Standard-Betriebssysteme,
	description={Ein Betriebssystem mit nennenswerter Marktbeteiligung, das zum Zeitpunkt der Entwicklung der Software nicht älter al 5 Jahre ist},
}
\newglossaryentry{Standard-Browser}
{
	name=Standard-Browser,
	description={Ein Browser mit nennenswerter Marktbeteiligung, das zum Zeitpunkt der Entwicklung der Software nicht älter als 5 Jahre ist}
}
\newglossaryentry{Sidebar}
{
	name=Sidebar,
	plural=Sidebars,
	description={Ein schmaler Bereich an der Seite einer Benutzeroberfläche, der funktionsspezifische Informationen und Funktionen bereit stellt}
}
\newglossaryentry{SensorThings API}
{
	name=SensorThings \gls{API},
	description={Standardisierte \gls{API} zur Abfrage von \gls{Sensor}daten aus einer Datenbank}
}
\newglossaryentry{RESTAPI}
{
	name=REST-\gls{API},
	description={REST steht für REpresentational State Transfer, API für Application Programming Interface. Gemeint ist damit ein Programmierschnittstelle, die sich an den Paradigmen und Verhalten des World Wide Web (WWW) orientiert und einen Ansatz für die Kommunikation zwischen Client und Server in Netzwerken beschreibt}
}
\newglossaryentry{API}
{
	name=API,
	description={Abkürzung für Application Programming Interface (engl.) bezeichnet eine Schnittstelle zu anderen Komponenten oder Programmen in der Softwarearchitektur}
}
\newglossaryentry{JavaScript}
{
	name=Java-Script,
	description={Java-Script ist eine Scriptsprache zur Ausführung im Webbrowser. Ursprünglich wurde Java-Script für das Laden von Seiten und Interagieren mit Nutzereingaben entwickelt. Heute findet Java-Script auch in größeren Strukturen Anwendung}
}
\newglossaryentry{Toolbar}
{
	name=Toolbar,
	description={Als Toolbar wird eine Aneinanderreihung von verschiedenen Werkzeugen bezeichnet. Die Toolbar dient dazu das ein Nutzer verschiedene Werkzeuge auswählen kann}
}
\newglossaryentry{MVC}
{
	name=MVC,
	description={Das Entwurfsmuster Model View Controller unterteilt Software in drei Bereiche. Durch die Gliederung in diese Bereiche werden spätere Änderungen an der Software vereinfacht und die Struktur verbessert}
}
\newglossaryentry{UML}
{
	name=UML,
	description={Die Unified Modeling Language, kurz UML, erlaubt das grafische Modellieren und Dokumentieren von Software}
}
\newglossaryentry{Luftverschmutzung}
{
	name=Luftverschmutzung,
	description={Als Luftverschmutzung bezeichnet man eine Verunreinigung oder Belastung der Umgebungsluft mit Schadstoffen. Solche Schadstoffe können beispielsweise  Rauch, Ruß, Staub, Abgase, Aerosole, Dämpfe und Geruchsstoffe sein}
}
\newglossaryentry{JSON}
{
	name=JSON,
	description={JavaScript Object Notationist ein kompaktes Datenformat in einer einfach lesbaren Textform und dient dem Zweck des Datenaustausches zwischen Anwendungen}
}