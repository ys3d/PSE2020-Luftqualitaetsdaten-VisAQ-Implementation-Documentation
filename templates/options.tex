\title{Implementierungsphase des PSE-Projekt}
\subtitle{Anwenderorientierte Nutzerschnittstelle für Luftqualitätsdaten Team 2}
\date{\today}
\author{Katharina Biernacka \and Maria Kraus \and Fabian Reinbold \and Daniel Schild}
\newcommand{\softwarename}{VisAQ }
\newcommand{\longSoftwarename}{Visualizing Air Quality}
\newcommand{\class}[1]{\paragraph{#1}\mbox{}\\}
\newcommand{\noChange}{Keine Änderungen}
\newcommand{\changedFunctions}{Folgende Funktionen wurden in ihrer Funktionalität geändert}
\newcommand{\removedClass}{Diese Klasse wurde aufgrund von React entfernt, da wir React-Lifecycle benutzen}
\newcommand{\toJSON}{Es wurde lediglich eine toJson() Methode hinzugefügt, die das Object in Json umwandelt.}
\newcommand{\controllerWrapper}{Aufgrund der JSON-Problematik wurden alle Controllerfunktionen mit Wrappern als Parameter ausgestattet. Die Funktionalität wurde in gleichnamige Funktionen mit den Ursprünglichen Parametern ausgelagert.}

\colorlet{punct}{red!60!black}
\definecolor{background}{HTML}{EEEEEE}
\definecolor{delim}{RGB}{20,105,176}
\colorlet{numb}{magenta!60!black}
\lstdefinelanguage{json}{
    basicstyle=\normalfont\ttfamily,
    numbers=left,
    numberstyle=\scriptsize,
    stepnumber=1,
    numbersep=8pt,
    showstringspaces=false,
    breaklines=true,
    frame=lines,
    backgroundcolor=\color{background},
    literate=
     *{0}{{{\color{numb}0}}}{1}
      {1}{{{\color{numb}1}}}{1}
      {2}{{{\color{numb}2}}}{1}
      {3}{{{\color{numb}3}}}{1}
      {4}{{{\color{numb}4}}}{1}
      {5}{{{\color{numb}5}}}{1}
      {6}{{{\color{numb}6}}}{1}
      {7}{{{\color{numb}7}}}{1}
      {8}{{{\color{numb}8}}}{1}
      {9}{{{\color{numb}9}}}{1}
      {:}{{{\color{punct}{:}}}}{1}
      {,}{{{\color{punct}{,}}}}{1}
      {\{}{{{\color{delim}{\{}}}}{1}
      {\}}{{{\color{delim}{\}}}}}{1}
      {[}{{{\color{delim}{[}}}}{1}
      {]}{{{\color{delim}{]}}}}{1},
}