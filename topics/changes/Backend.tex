\subsection{Backend}
Wir betrachten alle Klassen des Backends hinsichtlich den Schwierigkeiten bei der Implementierung und den Änderungen gegenüber dem Entwurf.

\subsubsection{Model}
\class{Datastream}
%TODO
\class{FeatureOfInterest}
\noChange
\class{HistoricalLocation}
\noChange
\class{Location}
\noChange
\class{Observation}
\noChange
\class{ObservedProperty}
%TODO
\class{Sensor}
%TODO
\class{Sensorthing}
\changedFunctions
\begin{itemize}[noitemsep]
    \item Die Java-Standart Funktion \textit{equals(Object other)} wurde aus den von Sensothing erbenden Klassen in die Klasse Sensorthing verschoben.
\end{itemize}
\class{SensorthingsProperties}
\noChange
\class{SensorthingsTimeStamp}
\noChange
\class{Thing}
%TODO
\class{UnitOfMeasurement}
\noChange
\class{PointDatum}
\noChange
\class{Sqaure}
Die Klasse Square wurde neu hinzugefügt um quadratische Bereiche in der Schnittstelle besser zu kapseln. Sqaure erbt von der ursprünglich verwendten Klasse Envelope.

\subsubsection{Controller}
%%%%%%%%%%%%%%%%%%%%%%%%%%%%%%%%%%%%%%%Links
\class{MultiLocalLink}
\noChange
\class{MultiNavigationLink}
Die Klasse wurde um eine geschachtetlte Klasse Builder erweitert die hilft einen MultiNavigationLinks des Online- beziehungsweise Offlinetyps zu bauen.
\class{MultiOnlineLink}
\noChange
\class{NavigationLink}
\noChange
\class{SingleLocalLink}
\noChange
\class{SingleNavigationLink}
Die Klasse wurde um eine geschachtetlte Klasse Builder erweitert die hilft einen SingleNavigationLinks des Online- beziehungsweise Offlinetyps zu bauen.
\class{SingleOnlineLink}
\noChange
%%%%%%%%%%%%%%%%%%%%%%%%%%%%%%%%%%%%%%%Controller
\class{DatastreamController}
\controllerWrapper
\\
Es wurden folgende Wrapper hinzugefügt
\begin{lstlisting}[language=java]
    static class ThingAndObservedPropertyWrapper
\end{lstlisting}
zum Kapseln eines Things mit einem ObservedProperty
und 
\begin{lstlisting}[language=java]
    static class SensorAndObservedPropertyWrapper
\end{lstlisting}
zum Kapseln eines Sensors und einem ObservedProperty
\class{FeatureOfInterestController}
\controllerWrapper
\class{HistoricalLocationController}
\controllerWrapper
\class{LocationController}
\controllerWrapper
\class{ObservationController}
\controllerWrapper
\\
Es folgende Wrapper hinzugefügt
\begin{lstlisting}[language=java]
    static class AreaWrapper
\end{lstlisting}
zum Kapseln eines Squares, einer Zeitspanne, eines Zeitpunktes und eines ObservedPropertys
\begin{lstlisting}[language=java]
    static class TimeframedThingWrapper
\end{lstlisting}
zum Kapseln mehrerer Things, eines Zeitpunktes, einer Zeitspanne und eines ObservedPropertys
\begin{lstlisting}[language=java]
    static class TopWrapper
\end{lstlisting}
zum Kapseln eines Datastreams und einer Anzahl an Elementen
\class{ObservedPropertyController}
\controllerWrapper
\class{SensorController}
\controllerWrapper
\class{SensorthingController}
Es wurde der Wrapper
\begin{lstlisting}[language=java]
    static class IdWrapper
\end{lstlisting}
hinzugefügt, der eine ID als String kapselt.
Der IDWrapper befindet sich in der Klasse SensorthingController, da alle Subklassen diesen Warpper benutzen.
Somit wird redundanz vermieden.
\class{ThingController}
\controllerWrapper
\class{UtilityController}
\controllerWrapper
\class{WebUtilityController}
Der WebUtilityController wurde neu hinzugefügt. Die hauptsächliche Verwendung liegt in der Anzeige der Versionsnummer aus der Klasse VisAQ mit einer GET-Request auf /version

Hierfür wurde die Methode
\begin{lstlisting}[language=java]
    /**
     * Returns the server-software-version.
     * 
     * @return The software-version
     */
    public String versionPage() {[...]}
\end{lstlisting}
implementiert
%%%%%%%%%%%%%%%%%%%%%%%%%%%%%%%%%%%%%%%Interpolation
\class{DefaultInterpolation}
%TODO
\class{Interpolation}
%TODO



\subsubsection{Sonstige}
\class{DedupingResolver}
Die Klasse DedupingResolver ist gegenüber dem Entwurf neu hinzugekommen.
\\
Die Klasse behebt Serialisierungsprobleme bei der \gls{JSON}-Serialisierung.

Betrachten wir hierzu folgendes vereinfachtes Beispiel im \gls{JSON}-Format:
\begin{lstlisting}[frame=single, language=json]
{
    "title": "Harry Potter and the Philosopher's Stone",
    "authors": [
        {
            "name": "JK Rowling"
        }
    ]
},
{
    "title": "Harry Potter and the Chamber of Secrets",
    "authors": [
        {
            "name": "JK Rowling"
        }
    ]
}
\end{lstlisting}
Hierbei repräsentiert \textit{authors} eine Hashmap.
Das Serialisierungsproblem besteht hierbei darin, dass das Untergeordnete Attribut \textit{authors} in einer einfachen Serialisierung doppelt vorkommt obwohl es sich um verschiedene Datensätze handelt.
Dieses Problem wird durch den DedupingResolver gelöst, indem Hashmaps besser serialisiert werden.
Hierfür erweitert er die Klasse SimpleObjectIdResolver, der sonst hierfür genutzt wird.
\class{RestConstants}
\noChange
\class{VisAQ}
\begin{itemize}[noitemsep]
    \item Ein Feld \textit{public static final String VERSION} wurde der klasse neu hinzugefügt. Die Variable soll nach dem Schema "yyyy.mm.dd\#x" gefüllt werden, wobei der erste Teil das aktuelle datum und x eine fortlaufende Nummer des Tages darstellt. Die Variable wird genutzt um bei aktiven Backends die aktuell arbeitende Version anzuzeigen.
\end{itemize}