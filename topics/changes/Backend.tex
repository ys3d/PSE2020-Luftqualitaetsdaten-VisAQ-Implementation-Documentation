\subsection{Backend}
Wir betrachten alle Klassen des Backends hinsichtlich den Schwierigkeiten bei der Implementierung und den Änderungen gegenüber dem Entwurf.

\subsubsection{Model}
\class{Datastream}
%TODO
\class{FeatureOfInterest}
\noChange
\class{HistoricalLocation}
\noChange
\class{Location}
\noChange
\class{Observation}
\noChange
\class{ObservedProperty}
%TODO
\class{Sensor}
%TODO
\class{Sensorthing}
\begin{itemize}[noitemsep]
    \item Die Java-Standart Funktion \textit{equals(Object other)} wurde aus den von Sensothing erbenden Klassen in die Klasse Sensorthing verschoben.
\end{itemize}
\class{SensorthingsProperties}
\noChange
\class{SensorthingsTimeStamp}
\noChange
\class{Thing}
%TODO
\class{UnitOfMeasurement}
\noChange
\class{PointDatum}
\noChange
\class{Sqaure}
Die Klasse Square wurde neu hinzugefügt um quadratische Bereiche in der Schnittstelle besser zu kapseln. Sqaure erbt von der ursprünglich verwendten Klasse Envelope.

\subsubsection{Controller}
%%%%%%%%%%%%%%%%%%%%%%%%%%%%%%%%%%%%%%%Links
\class{MultiLocalLink}
%TODO
\class{MultiNavigationLink}
%TODO
\class{MultiOnlineLink}
%TODO
\class{NavigationLink}
%TODO
\class{SingleLocalLink}
%TODO
\class{SingleNavigationLink}
%TODO
\class{SingleOnlineLink}
%TODO
%%%%%%%%%%%%%%%%%%%%%%%%%%%%%%%%%%%%%%%Controller
\class{DatastreamController}
%TODO
\class{FeatureOfInterestController}
%TODO
\class{HistoricalLocationController}
%TODO
\class{LocationController}
%TODO
\class{ObservationController}
%TODO
\class{ObservedPropertyController}
%TODO
\class{SensorController}
%TODO
\class{SensorthingController}
%TODO
\class{ThingController}
%TODO
\class{UtilityController}
%TODO
\class{WebUtilityController}
%TODO
%%%%%%%%%%%%%%%%%%%%%%%%%%%%%%%%%%%%%%%Interpolation
\class{DefaultInterpolation}
%TODO
\class{Interpolation}
%TODO



\subsubsection{Sonstige}
\class{DedupingResolver}
Die Klasse DedupingResolver ist gegenüber dem Entwurf neu hinzugekommen. Sie wird verwendte um %TODO
\class{RestConstants}
\noChange
\class{VisAQ}
\begin{itemize}[noitemsep]
    \item Ein Feld \textit{public static final String VERSION} wurde der klasse neu hinzugefügt. Die Variable soll nach dem Schema "yyyy.mm.dd\#x" gefüllt werden, wobei der erste Teil das aktuelle datum und x eine fortlaufende Nummer des Tages darstellt. Die Variable wird genutzt um bei aktiven Backends die aktuell arbeitende Version anzuzeigen.
\end{itemize}