\subsection{Frontend}
Im Frontend betrachten wir Änderungen nicht Funktionsweise, vielmehr betrachten wir die Umsetzung der Klassen des Entwurfs im praktischen React-Programm.
\subsubsection{Model}
\class{DataStream}
\class{FeatureOfInterest}
\class{HistoricalLocation}
\class{Location}
\class{Observation}
\class{ObservedProperty}
\class{PointDatum}
\class{Sensor}
\class{SensorDatum}
\class{Sensorthing}
\class{Thing}
\class{UnitOfMeasurement}

\subsubsection{View}
\class{ExpertViewFilter}
\class{Navbar}
    Umbenannt in Navigationbar, da es in React-Bootstrap schon eine implementierte Klasse Navbar gibt.
\class{Toolbar}
    \removedClass
    Diese Klasse wird nun direkt in der Navigationbar mitimplementiert.
\class{ObservedNavbarSubject}
\class{Timeline}
\class{ColorBlindTheme}
\class{ColorTheme}
\class{DarkTheme}
\class{Gradient}
\class{LightTheme}
\class{CookieNotice}
    Diese Klasse zeigt die CookieNotice als Modal an, und speichert zugleich bei Akzeptanz durch den Benutzer die Cookies.
    Diese beinhalten die gewählte Sprache und den aktuellen Standort des Benutzers.
\class{InformationView}
\class{Language}
    Anstatt eine eigene Language implementierung zu schreiben benutzen wir die zwei Dependencies i18n und i18next.
    Diese werden in der Language Klasse initialisiert. 
    Außerdem wird in der componentWillMount() Methode geprüft ob Cookies existieren und daraufhin wird die Sprache angepasst. Dies geschieht über die changeLanguage() von i18next und über die Methode getLanguage().
    Diese Methode spaltet die Cookie-Json file auf und gibt die gespeicherte Sprache zurück.
\class{MapView}
\class{View}
    \removedClass
\class{InterpolationOverlayFactory}
\class{OverlayBuilder}
\class{OverlayFactory}
\class{SensorOverlayFactory}
\class{SensorOverview}
    Diese Klasse 
\class{Legend}
    Die Klasse implementiert die angezeigte Legende, welche mithilfe von leaflet und unserer Gradient Klasse erstellt wird.
    \standardMethods
    Mit der Methode createLegend() wird eine Legende erstellt und der Karte hinzugefügt.
    Mit der Methode removeLegend() wird die Legende entfernt
\class{Diagram}
    Die Klasse implementiert ein Liniendiagramm für die Sidebar. Genutzt wird hierbei die React-chartjs-2 Dependency.
    Mithilfe dieser kann man Diagramme auf den gegebenen Daten konfigurieren. Diese werden mithilfe der componentDidUpdate(prevProps) Methode
    bei einer Änderung der Daten erneuert.
\class{LineDiagram}
    \removedClass
\class{BarDiagram}
    \removedClass
\class{AirQualityData}

Neu dazu kommt:
\class{DataCard}
    Eine rein graphische Klasse welche dafür sorgt, dass die Diagramme in dem Akkordion der SensorOverview angezeigt werden.
\class{OverviewContainer}
\class{PointOverview}
\class{ShareField}
    Diese Klasse implementiert mithilfe der react-share Dependency ShareButtons für Email, Reddit, WhatApp und Telegram.
    Hierbei wird die default Klasse exportiert, welche die render() Methode beinhaltet und darin die einzelnen Buttons deklariert und 
    mit der passenden URL verbindet.
\class{PopupCauses}
    Die Klasse beinhaltet das Popup welches die Konsequenzen von hohen Feinstaub anzeigt. Hierbei handelt es sich um keine funktionale Klasse sondern um eine Klasse die sich
    lediglich um graphische Komponenten kümmert.
\class{PopupReasons}
    Die Klasse beinhaltet das Popup welches die Gründe von hohen Feinstaub anzeigt. Hierbei handelt es sich um keine funktionale Klasse sondern um eine Klasse die sich
    lediglich um graphische Komponenten kümmert.
\class{CookieNoticeInformation}
    Die Klasse beinhaltet das Popup welches die Erklärung für unsere Nutzung von Cookies anzeigt. Hierbei handelt es sich um keine funktionale Klasse sondern um eine Klasse die sich
    lediglich um graphische Komponenten kümmert.
\class{App}
    Die Klasse App ersetzt
\subsubsection{Controller}
\class{Request}
    Diese Klasse stellt eine POST Request an das Backend.