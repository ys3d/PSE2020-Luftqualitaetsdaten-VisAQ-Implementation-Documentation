\subsection{Backend}
\subsubsection{Test-Konzept}
Im Backend werden JUnit\footnote{\url{https://junit.org/}} Tests verwendet, es werden hierbei Tests für die JUnit Version 5 geschrieben.
Um die Qualität und die Code-Abdeckung der Testfälle zu beurteilen wird das System Pitest\footnote{\url{https://pitest.org/}} verwendet.
Pitest führt Mutation Testing\footnote{\url{https://en.wikipedia.org/wiki/Mutation_testing}} durch.
Bei dieser Methode werden Testfälle getestet, indem der Programmcode an je einer Stelle pro Durchlauf verändert wird.
Nun sollte mindestens einer der Testfälle fehlschlagen.
Ist dies nicht der Fall, so wurde die veränderte Codezeile nicht hinreichend genau getestet.
Zusätzlich generiert Pitest einen umfangreichen Testbericht und erfasst auch die Reine Zeilenüberdeckung der Tests.

\subsubsection{Testen des Modells}
Im Modell werden alle Klassen getestet.
\\
Getestet wird mit 100\% Code-Überdeckung getestet

Für jede Klasse des Modells wurde eine Test-Klasse erstellt die dafür gedacht ist ausschließlich das Verhalten der entsprechenden Klasse zu testen.

\subsubsection{Testen des Controllers}
