\section{Zielbestimmungen}
\subsection{Musskriterien}
    Hierbei handelt es sich um unsere umgesetzten Musskriterien aus dem Pflichtenheft.
    \begin{tabularx}{\textwidth}{| X | X |}
    \hline
    \textbf{Kriterium} & 
    \textbf{Implementiert}\\
    \hline
    Der Nutzer kann auf die Webapplikation über einen Webbrowser zugreifen. Der Zugriff soll sowohl über den Computer als auch über Handys möglich sein. & Wurde so umgesetzt \\
    \hline
    Als erste Seite soll eine Landkarte geladen werden. Diese Karte soll Aufschluss über die \gls{Feinstaub}messwerte zu einem in der näheren Vergangenheit gemessenen Zeitpunkt geben. & Wurde so umgesetzt \\
    \hline
    Der Nutzer soll auf einer Toolbar auswählen können welche \gls{Luftqualitaetsdaten} auf der Landkarte angezeigt werden. Der Nutzer hierbei die Auswahl zwischen den \gls{Luftqualitaetsdaten} Luftfeuchtigkeit, Luftdruck, Temperatur und \gls{Feinstaub}. & Wurde so umgesetzt \\
    \hline
    Die unterschiedlichen \gls{Luftqualitaetsdaten} sollen mit farbigen Kartenoverlays auf der Karte dargestellt werden. Dabei soll für Temperatur- und \gls{Feinstaub}werte eine rot-grün Skala und Luftdruck und Luftfeuchtigkeit in einer gelb-blau Skala verwendet werden. \gls{Feinstaub}messwerte sollen dabei hervorgehoben werden falls sie die gesetzlich vorgeschriebenen Grenzwerte überschreiten.
    Es gibt eine Legende für die verwendeten Farben. & Wurde so umgesetzt, jedoch haben sich die Farbskalen den gängigen Farben für die Werte angepasst. \\
    \hline
    Die Messwerte sollen durch interpolierte Daten kontinuierlich dargestellt werden. & Wurde so umgesetzt \\
    \hline
    Der Nutzer kann auf einen Sensor markieren und bekommt dessen Messwerte in einem Diagramm und den Sensortyp angezeigt. & Anstatt einem Diagram wird für jeden Messwerttyp ein eigenes Diagramm ausgegeben \\
    \hline
    Der Nutzer kann auf einen Punkt auf der Landkarte markieren und bekommt dessen interpolierte Messwerte als Diagramm angezeigt. & Es werden lediglich die aktuellen Daten des Ortes angezeigt \\
    \hline
    Der Nutzer kann eine Stadt durch den Stadtnamen oder die Postleitzahl suchen. Die Webapplikation ändert daraufhin den Landkartenausschnitt und markiert die gesuchte Stadt auf der Karte. & Die Suchfunktion wurde implementiert, jedoch ohne die Markier-Funktion, vielmehr wird zu dem gesuchten Ort gezoomt \\
    \hline
    Zu einem Sensor der einem Punkt auf der Landkarte kann der Nutzer ein Zeitintervall angeben. Die Webapplikation zeigt daraufhin für das gegebenen Intervall die zeitliche Entwicklung der Luftqualität als Menge-Zeit Diagramm an. & Wurde nicht umgesetzt \\
    \hline
    Zu einem Sensor oder einem Punkt auf der Landkarte kann der Nutzer ein Datum filtern. Die Webapplikation zeigt daraufhin die entsprechenden Messwerte zur Luftqualität an. & Diese Funktion wurde in Form der Diagramme umgesetzt \\
    \hline
    Auf der Webapplikation soll ebenfalls eine Zeitachse dargestellt werden. Der Nutzer kann hier durch das Verschieben eines Reglers frühere Messwerte zu Luftqualität abrufen.  & Wurde nicht umgesetzt \\
    \hline
    Die Webapplikation enthält eine Definition für \gls{Feinstaub}. & Wurde so umgesetzt \\
    \hline
    Die Webapplikation enthält eine Verlinkung zum Projekt SmartAQnet. & Wurde so umgesetzt \\   
    \hline 
    \end{tabularx}
\subsection{Wunschkriterien}
    Hierbei handelt es sich um unsere umgesetzten Wunschkriterien aus dem Pflichtenheft.
    
    \begin{tabularx}{\textwidth}{| X | X |}
    \hline
    \textbf{Kriterium} & 
    \textbf{Implementierung}\\
    \hline
    Die Webapplikation soll die Hauptgründe für \gls{Feinstaub} in Deutschland in einer Graphik darstellen. & Anstelle einer Graphik gibt es einen Fließtext \\
    \hline
    Die Webapplikation soll die Gesundheitsrisiken, die durch \gls{Feinstaub} auftreten, in einer Graphik darstellen. & Anstelle einer Graphik gibt es einen Fließtext \\
    \hline
    Der Nutzer kann den \gls{Feinstaub}messwert eines Sensors oder eines Punkts auf der Karte auswählen. 
    Die Webapplikation zeigt daraufhin an welche Gesundheitsrisiken bei diesem spezifischen Messwert kurzfristig und langfristig auftreten. & Wurde nicht umgesetzt \\
    \hline
    Der Nutzer kann den \gls{Feinstaub}messwert eines Sensors oder eines Punkts auf der Karte auswählen. 
    Die Webapplikation zeigt ihm daraufhin ein alltägliches Szenario an, das eine äquivalente \gls{Feinstaub}belastung verursacht. & Wurde nicht umgesetzt \\
    \hline
    Auf Wunsch des Nutzers soll die Webapplikation dessen Standort feststellen und die Karte entsprechend zentrieren. & Wurde nicht umgesetzt \\
    \hline
    Durch eine Teilen-Funktion soll der Nutzer Messwerte in den sozialen Netzwerken verbreiten können. & Wurde so umgesetzt \\
    \hline
    Beim Verlassen der Webapplikation soll der zuletzt betrachtete Kartenausschnitt gespeichert werden. Beim erneuten Betreten der Seite soll dieser Kartenausschnitt wiederhergestellt werden. & Die Seite zentriert beim erneuten Besuchen und bei akzeptierten Cookies auf den aktuellen Standort des Benutzers \\
    \hline
    Die Webapplikation soll eine Hilfefunktion haben, die die verschiedenen Interaktionsmöglichkeiten erklärt. & Wurde nicht umgesetzt \\
    \hline
    Die Webapplikation enthält einen Link zu einer Bauanleitung für einen \gls{DIY}-Sensor. & Wurde so umgesetzt \\
    \hline
    Der Nutzer kann auswählen, ob ihm die Webapplikation in deutscher oder englischer Sprache angezeigt wird. & Wurde so umgesetzt \\
    \hline
    Der Nutzer soll, als zusätzliche Anzeige, die Webapplikation in einem Dark-Mode oder in einem Farbenblind-Modus benutzen können. & Wurde nicht umgesetzt \\
    \hline
    Die Webapplikation verfügt über einen Expertenmodus. In diesem werden, wenn der Nutzer auf einen Sensor klickt, typspezifische Informationen über den Sensor angezeigt.
    Im Expertenmodus existiert ein Filter über die Sensoren. Der Nutzer kann einstellen welche Sensor-Typen auf der Landkarte angezeigt werden. & Der Expertenmodus zeigt erweiterte Informationen in der SensorOverview an\\
    \hline
    \end{tabularx}