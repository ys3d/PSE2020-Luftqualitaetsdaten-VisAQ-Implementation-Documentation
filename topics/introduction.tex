% Einleitung
\section{Einleitung}
VisAQ ist eine Webanwendung um Luftqualitätsdaten eines heterogenen \gls{Sensor}netzwerks für eine breite Nutzergruppe zu Visualisieren.
Dafür wird auf die Daten des Projektes SmartAQNet\footnote{\url{https://www.smartaq.net}} zurück gegriffen.
SmartAQNet sammelt Sensordaten aus dem Großraum Augsburg und bündelt diese in einer Datenbank nach dem Sensorthings-Standart\footnote{\url{https://developers.sensorup.com/docs/}}.
\\
Das Projekt wird im Rahmen des Softwareprojektes PSE am Karlsruher Institut für Technologie durchgeführt.
\\
Auf Basis des Pflichtenhefts\footnote{\url{https://github.com/ys3d/PSE2020-Luftqualitaetsdaten-Pflichtenheft}} und des Entwurf\footnote{\url{https://github.com/ys3d/PSE2020-Luftqualitaetsdaten-Entwurf}} wurde VisAQ implementiert.
\\
